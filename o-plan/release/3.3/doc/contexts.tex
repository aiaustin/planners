\documentstyle [a4]{article}

% This is a draft of a section of the O-Plan Implementation Guide

\raggedbottom

% Some commands for O-Plan source/doc/*.tex files.

% Author: Jeff Dalton

% Updated: Tue Feb 18 06:08:44 1997 by Jeff Dalton


% Commands for BNF, etc.

\newcommand{\var}[1]{\mbox{\em #1\/}}
\newcommand{\meta}[1]{{\rm $\langle$#1\/$\rangle$}}
% \newcommand{\meta}[1]{{$\langle$\var{#1}$\rangle$}}
\newcommand{\arbno}[1]{#1{$\;\ldots$}} % ie, zero or more
\newcommand{\optional}[1]{{\rm [}#1{\rm ]}}
\newcommand{\evalsto}{{$\Rightarrow\;$}}
\newcommand{\equivto}{{$\equiv\;\,$}}
\newcommand{\expandsto}{{$\Longrightarrow\;$}}
\newcommand{\MapsTo}{{$\rightarrow\;\,$}}
\newcommand{\BNFdef}{{\rm ::=\ }}
\newcommand{\ordef}{$\;\mid\;$}

\newcommand{\bool}{{\it true\/} {\rm or} {\it false}}

\newcommand{\listof}[1]{\mbox{{\rm list of #1}}}

% The @{} in the "cols" of the tabular env eliminates the extra space
% before the first column and after the last.  (See LaTeX section C.9.2)

\newenvironment{syntaxtable}{
   \begin{tt}
   \begin{trivlist}
           \item[]
   \begin{tabular}{@{}lll@{}}
 }{\end{tabular}
   \end{trivlist}
   \end{tt}}


% Program text may be given in a verbatim environment or, when some parts
% should not be in the typewriter font, in a tabbing environment.  Progtab
% is intended for the latter case.  It makes tt the default font.

% progt is similar to progtab but indents on the left.  Perhaps better
% names could be chosen.  ttab?

\newenvironment{progtab}{
   \begin{tt}
   \begin{tabbing}
 }{\end{tabbing}
   \end{tt}}

\newenvironment{progt}{
   \begin{Lindent}
   \begin{tt}
   \begin{tabbing}
 }{\end{tabbing}
   \end{tt}
   \end{Lindent}}


% Lindent is indented on the left as if in a \quote, but without
% having a similar indent on the right.  It therefore has more room
% before it's ``too wide''.

\newenvironment{Lindent}{
   \begin{tt}
   \begin{list}{}%
               {}
               \item[]
 }{\end{list}
   \end{tt}}


% Function, variable, and other definitions can be provided by a
% lispdef environment.  It produces a one-line "syntax" specification
% followed by an indented block similar to the Lindent block defined above.
% For example,
%
%   \begin{lispdef}{(f \var{x})}{Function}
%      Applies {\tt f} to \var{x}.
%   \end{lispdef}
%
% would produce something roughly like this:
%
%   (f x)                     [Function]
%
%      Applies f to x.
%
% "f" would be in tt font, "x" would be italic (since that's what \var does).
% The "[Function]" would be out at the right margin.
%
% The lispdef env is used to define other envs; it is not normally
% used directly.  For instance, the above example would normally be
% written using a defun env, as defined below.
%
% \parskip is set to 0 to undo the usual way of writing a document
% with unindented paragraphs: setting \parindent to 0pt and \parskip
% to, say, 8pt.
%

\newenvironment{lispdef}[2]{
   \begin{tabbing}
     {\tt #1} \` [{\it #2\/}]
   \end{tabbing}
   \begin{list}{}%
         {
%         \setlength{\leftmargin}{\parindent}
%         \setlength{\rightmargin}{\leftmargin}
%         \setlength{\leftmargin}{0pt}
          \setlength{\parskip}{0pt}
         }
         \item[]
         \vspace{-\topsep}
 }{\end{list}}

% Defining variables and parameters.
%   #1 = the variable name

\newenvironment{defvar}[1]{
   \begin{lispdef}{#1\index{#1}}{Variable}
 }{\end{lispdef}}

\newenvironment{defparameter}[1]{
   \begin{lispdef}{#1\index{#1}}{Parameter}
 }{\end{lispdef}}

% Defining functions, macros, and messages
%   #1 = the function, macro, or message name
%   #2 = a description of the arguments.
% The argument description should begin with a space to separate it
% from #1.  We can't add the space automatically, because sometimes
% the argument description should be null.  #1 is a spearate argument
% so it can be indexed.

\newenvironment{defun}[2]{
   \begin{lispdef}{(#1\index{#1}#2)}{Function}
 }{\end{lispdef}}

\newenvironment{defmacro}[2]{
   \begin{lispdef}{(#1\index{#1}#2)}{Macro}
 }{\end{lispdef}}

\newenvironment{defmessage}[2]{
   \begin{lispdef}{(#1\index{#1}#2)}{Message}
 }{\end{lispdef}}


% End


\begin{document}

\section{Context-layering}

\subsection{Introduction}

An access function, or {\em accessor}, is a function that can be
applied to an object to obtain a value (another object), typically a
value stored in a field or {\em slot}.  For instance, {\tt defstruct}
defines an accessor for each slot in a structure.

In O-Plan, a {\em context\/} is an object that is used as an
additional, implicit parameter to an accessor.  An ordinary accessor is
a mapping from objects to values; a context-layered or context-relative
accessor is a mapping from objects {\em and the current context\/} to
values.  To make the presentation simpler, we will sometimes talk as if
context-relative values were always stored in slots.  However, layered
accessors do not have to be based on slot accessors; they can be based
on any mapping from objects to values.

A context has few properties of its own.  It is simply an object with
which the values considered by context-relative accessors can be
associated.  Contexts are arranged in a tree: each context has a
single predecessor (parent) and a list of successors (children).  
When there is no value directly associated with a context, a value is
inherited from the nearest ancestor context that does have a value
associated with it.  However, when a value is assigned in a context,
it is always associated directly with that context; the associations
between values and ancestor contexts are not changed.  This makes it
possible to protect the values associated with a context, while
still being able to access them, by moving to a child context.

One context is always designated as the current context, and only
values associated with that context or its ancestors are normally
seen.

The O-Plan context-layering mechanism provides a way to define
context-relative accessors based on ordinary access functions.
If $\cal{F}$ is the name of an accessor such that 
\begin{itemize}
\item {\tt ($\cal{F}$ \var{x})} accesses a slot of (or location in)
      objects \var{x} of a certain type, and
\item {\tt (setf ($\cal{F}$ \var{x}) \var{value})} sets the same slot,
\end{itemize}
then a context-relative version of $\cal{F}$ named $\cal{C}$ can be
defined by

\begin{tabbing}
{\tt (define-context-accessor $\cal{C}$ $\cal{F}$)}
\end{tabbing}

\noindent
{\tt ($\cal{C}$ \var{x})} will then return the slot value associated
with the current context, directly or by inheritance, and {\tt setf}
can be used with $\cal{C}$ to directly associate a value with the
current context.

Note that the slots of an object are normally context-layered
independently.  This is an advantage when only a few slots are layered
or when the pattern of associations between slot values and contexts
is not uniform.  However, there are circumstances in which it would be
better to layer entire objects instead.  One way to handle this in
O-Plan be to define a context-layered mapping from object ids (names)
to objects and make a copy of an object (for the current context)
when the first slot update is done.  


\subsection{Internals}

Context-layered accessors are perfectly happy with ordinary slot
values such as might be installed when an object is first created.
However, when a context-relative value is first assigned (by using
{\tt setf} with a context-layered accessor), the ordinary value
is replaced by a {\tt cval} structure that contains a mapping from
contexts to (ordinary) values.

Consequently, a layered accessor normally works by:
\begin{enumerate}
\item calling an ordinary accessor to obtain a {\tt cval} structure,
      and
\item looking in the {\tt cval} structure for the value associated
      with the current context or with the nearest ancestor that has a
      value.
\end{enumerate}
Both steps can be performed by calling the function {\tt access-in-context}.
The second step, dereferencing a {\tt cval}, can be performed by
calling {\tt deref-in-context}.

The {\tt cval} structure's mapping from contexts to values is an
a-list.  However, we don't want to search the complete a-list for the
current context, and then search it again for each ancestor of the
current context, until we finally find an entry that matches.  Nor do
we want to go through the a-list entries once, repeating a search
along parent links from the current context each time (even if we
could still make inheritance work correctly that way).  Consequently,
we combine the traversal of the a-list with that of the parent links
in such a way that there's no repeated work.

An efficient search for the value associated with a context, or
inherited from an ancestor, is accomplished by exploiting context
numbers.  Every context has a number which is assigned when it is
created, and newer contexts have higher numbers than older ones.  In
particular, a context has a higher number than any of its ancestors.
The a-list entries in {\tt cval} structures are ordered by context
number, with higher-numbered (newer) contexts first.  The algorithm
for finding the right value in a {\tt cval} structure is therefore as 
follows:

\begin{enumerate}

\item Initialize \var{target} to the current context and \var{a-list}
      to the a-list in the {\tt cval} structure.

\item Discard entries for contexts newer than \var{target} from the
      front of \var{a-list}.  If \var{a-list} becomes null, return
      {\tt :undef}.

\item At this point, the first of the remaining entries in \var{a-list}
      should be for \var{target} or for a context older than \var{target}.
      If it's for \var{target}, then we're done: return the associated
      value.  Otherwise, set \var{target} to the parent of \var{target}
      and return to step 2.

\end{enumerate}

In addition, a {\tt cval} contains a one-context cache.  If the
current context is the same as the cached context, then the cached
value is returned.  Otherwise, the algorithm above is used to find a
value, and that value and the current context are cached.


\subsection{Externals}


\subsubsection{Defining forms}

\begin{tabbing}
{\tt (define-context-accessor \var{name} \var{accessor-name})}
\end{tabbing}

Defines \var{name} as a context-relative accessor based on 
\var{accessor-name}.


\subsubsection{Variables}

\begin{tabbing}
{\tt *context*}
\end{tabbing}

A variable whose value is the currently active context.  It's
possible to make arbitrary moves in the context tree by assigning
to or binding {\tt *context*}.

\begin{tabbing}
{\tt *global-context*}
\end{tabbing}

A variable whose value is a root context supplied by the system.  
When a location is first assigned a context-relative value by using
{\tt setf} with a context-relative accessor, any existing value is
made the value in {\tt *global-context*}.


\subsubsection{Basic context operations}

\begin{tabbing}
{\tt (new-context \var{parent})} \MapsTo \var{new child context}
\end{tabbing}

Creates a new context with parent \var{parent}.  \var{Parent} can be
{\tt nil} to create a new root context.

\begin{tabbing}
{\tt (context-parent \var{context})} \MapsTo \var{parent context or {\tt nil}}
\end{tabbing}

Returns the predecessor (parent) of a context.

\begin{tabbing}
{\tt (contexts-children \var{context})} \MapsTo \var{list of child contexts}
\end{tabbing}

Returns a list of the successors (children) of a context.

\begin{tabbing}
{\tt (context-number \var{context})} \MapsTo \var{context-number}
\end{tabbing}

Returns the number of a context.

If {\tt (context-number \var{c})} returns \var{i}, 
then {\tt (find-context \var{i})} returns \var{c}.

\begin{tabbing}
{\tt (find-context \var{context-number})} \MapsTo \var{context}
\end{tabbing}

Returns the context that has the given number.

If {\tt (find-context \var{i})} returns \var{c},
then {\tt (context-number \var{c})} returns \var{i}.

\begin{tabbing}
{\tt (push-context)} \MapsTo \var{new child context}
\end{tabbing}

Creates a new child of {\tt *context*} and then sets {\tt *context*}
to be that child.

\begin{tabbing}
{\tt (pop-context)} \MapsTo \var{parent context or {\tt nil}}
\end{tabbing}

If {\tt *context*} is not a root context (i.e., if it has a non-null
parent), {\tt popcontext} sets {\tt *context*} to the parent of
{\tt *context*}.  Otherwise, {\tt pop-context} simply returns {\tt nil}
without changing the value of {\tt *context*}.

\begin{tabbing}
{\tt (in-context \var{context} \var{function} \&rest \var{arguments})}
     \MapsTo \var{value}
\end{tabbing}

Applies \var{function} to \var{arguments} with {\tt *context*}
temporarily bound to \var{context}.


\subsubsection{Value lookup and assignment}

\begin{tabbing}
{\tt (access-in-context \var{accessor} \var{object})} \MapsTo \var{value}
\end{tabbing}

Applies \var{accessor} to \var{object} to obtain either an ordinary
value or a {\tt cval} structure.  An ordinary value is simply
returned.  A {\tt cval} structure is searched to find the value
associated with {\tt *context*} or the nearest ancestor of
{\tt *context*}.  If such a value is found, it is returned;
otherwise, {\tt :undef} is returned.

\begin{tabbing}
{\tt (update-in-context \var{updater} \var{accessor} \var{object} 
                        \var{new-value})}
     \MapsTo \var{new-value}
\end{tabbing}

Establishes \var{new-value} as the value for {\tt *context*}
so that {\tt (access-in-context \var{accessor} \var{object})} will
find \var{new-value} as the value for {\tt *context*} directly
rather than by inheritance from an ancestor of {\tt *context*}. 
The values associated with such ancestors are not changed.

\var{Accessor} is applied to \var{object} to obtain either an ordinary
value or a {\tt cval} structure.  An ordinary value is replaced by a
{\tt cval} structure in which the ordinary value is associated with
{\tt *global-context*}.  Otherwise, the existing {\tt cval} structure is
modified.

\var{Updater} should set the same field that \var{accessor} accesses
so that after {\tt (funcall \var{updater} \var{object} \var{v})},
{\tt (funcall \var{accessor} \var{object})} will return \var{v}.

\begin{tabbing}
{\tt (deref-in-context \var{object})} \MapsTo \var{value}
\end{tabbing}

If \var{object} is not a {\tt cval} structure, then {\tt deref-in-context}
returns \var{object}; otherwise, it finds a value for the current context
in the same manner as {\tt access-in-context}.

\begin{tabbing}
{\tt (ctxt-symbol-value \var{symbol})} \MapsTo \var{value}
\end{tabbing}

A context-layered version of {\tt symbol-value} defined by:

\begin{verbatim}
(define-context-accessor ctxt-symbol-value symbol-value)
\end{verbatim}

\begin{tabbing}
{\tt (ctxt-gethash \var{key} \var{hash-table})}
   \MapsTo \var{value}, \bool \\
{\tt (ctxt-puthash \var{key} \var{hash-table} \var{value})}
   \MapsTo \var{value}
\end{tabbing}

A context-layered version of {\tt gethash} and an associated update
function.  {\tt ctxt-gethash} can also be used with {\tt setf}.

\subsubsection{Utilities}

\begin{tabbing}
{\tt (print-context-tree)}
\end{tabbing}

Prints the context numbers of all contexts with children listed
below their parent and indented further to the right.

\end{document}
