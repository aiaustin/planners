\documentstyle [a4]{article}

% This is a draft of a section of the O-Plan Implementation Guide

\raggedbottom

% Some commands for O-Plan source/doc/*.tex files.

% Author: Jeff Dalton

% Updated: Tue Feb 18 06:08:44 1997 by Jeff Dalton


% Commands for BNF, etc.

\newcommand{\var}[1]{\mbox{\em #1\/}}
\newcommand{\meta}[1]{{\rm $\langle$#1\/$\rangle$}}
% \newcommand{\meta}[1]{{$\langle$\var{#1}$\rangle$}}
\newcommand{\arbno}[1]{#1{$\;\ldots$}} % ie, zero or more
\newcommand{\optional}[1]{{\rm [}#1{\rm ]}}
\newcommand{\evalsto}{{$\Rightarrow\;$}}
\newcommand{\equivto}{{$\equiv\;\,$}}
\newcommand{\expandsto}{{$\Longrightarrow\;$}}
\newcommand{\MapsTo}{{$\rightarrow\;\,$}}
\newcommand{\BNFdef}{{\rm ::=\ }}
\newcommand{\ordef}{$\;\mid\;$}

\newcommand{\bool}{{\it true\/} {\rm or} {\it false}}

\newcommand{\listof}[1]{\mbox{{\rm list of #1}}}

% The @{} in the "cols" of the tabular env eliminates the extra space
% before the first column and after the last.  (See LaTeX section C.9.2)

\newenvironment{syntaxtable}{
   \begin{tt}
   \begin{trivlist}
           \item[]
   \begin{tabular}{@{}lll@{}}
 }{\end{tabular}
   \end{trivlist}
   \end{tt}}


% Program text may be given in a verbatim environment or, when some parts
% should not be in the typewriter font, in a tabbing environment.  Progtab
% is intended for the latter case.  It makes tt the default font.

% progt is similar to progtab but indents on the left.  Perhaps better
% names could be chosen.  ttab?

\newenvironment{progtab}{
   \begin{tt}
   \begin{tabbing}
 }{\end{tabbing}
   \end{tt}}

\newenvironment{progt}{
   \begin{Lindent}
   \begin{tt}
   \begin{tabbing}
 }{\end{tabbing}
   \end{tt}
   \end{Lindent}}


% Lindent is indented on the left as if in a \quote, but without
% having a similar indent on the right.  It therefore has more room
% before it's ``too wide''.

\newenvironment{Lindent}{
   \begin{tt}
   \begin{list}{}%
               {}
               \item[]
 }{\end{list}
   \end{tt}}


% Function, variable, and other definitions can be provided by a
% lispdef environment.  It produces a one-line "syntax" specification
% followed by an indented block similar to the Lindent block defined above.
% For example,
%
%   \begin{lispdef}{(f \var{x})}{Function}
%      Applies {\tt f} to \var{x}.
%   \end{lispdef}
%
% would produce something roughly like this:
%
%   (f x)                     [Function]
%
%      Applies f to x.
%
% "f" would be in tt font, "x" would be italic (since that's what \var does).
% The "[Function]" would be out at the right margin.
%
% The lispdef env is used to define other envs; it is not normally
% used directly.  For instance, the above example would normally be
% written using a defun env, as defined below.
%
% \parskip is set to 0 to undo the usual way of writing a document
% with unindented paragraphs: setting \parindent to 0pt and \parskip
% to, say, 8pt.
%

\newenvironment{lispdef}[2]{
   \begin{tabbing}
     {\tt #1} \` [{\it #2\/}]
   \end{tabbing}
   \begin{list}{}%
         {
%         \setlength{\leftmargin}{\parindent}
%         \setlength{\rightmargin}{\leftmargin}
%         \setlength{\leftmargin}{0pt}
          \setlength{\parskip}{0pt}
         }
         \item[]
         \vspace{-\topsep}
 }{\end{list}}

% Defining variables and parameters.
%   #1 = the variable name

\newenvironment{defvar}[1]{
   \begin{lispdef}{#1\index{#1}}{Variable}
 }{\end{lispdef}}

\newenvironment{defparameter}[1]{
   \begin{lispdef}{#1\index{#1}}{Parameter}
 }{\end{lispdef}}

% Defining functions, macros, and messages
%   #1 = the function, macro, or message name
%   #2 = a description of the arguments.
% The argument description should begin with a space to separate it
% from #1.  We can't add the space automatically, because sometimes
% the argument description should be null.  #1 is a spearate argument
% so it can be indexed.

\newenvironment{defun}[2]{
   \begin{lispdef}{(#1\index{#1}#2)}{Function}
 }{\end{lispdef}}

\newenvironment{defmacro}[2]{
   \begin{lispdef}{(#1\index{#1}#2)}{Macro}
 }{\end{lispdef}}

\newenvironment{defmessage}[2]{
   \begin{lispdef}{(#1\index{#1}#2)}{Message}
 }{\end{lispdef}}


% End


\begin{document}

\tableofcontents

\newpage

\section{Task Formalism compiler}

The TF Compiler processes a domain description written in the Task
Formalism language and returns a structures that contains the same
information in a more easily manipulated form.  It is usually called
by the database manager on behalf of a knowledge source (typically
KS-DOMAIN), but it can also be used as a syntax-checker without
involving the rest of O-Plan.

The compiler uses a fairly conventional recursive descent parser and
a scanner (or ``tokenizer'') that employs the Common Lisp readtable
mechanism.  In addition to parsing, it detects a number of errors and
inconsistencies that can occur in domain descriptions and performs a
certain amount of compile-time analysis, especially of action and
effect ``levels''.  The interface to the compiler and the main
components of the implementation are described below.

\subsection{Interface}

The TF Compiler is invoked by calling the function
{\tt compile-tf-stream}:

\begin{tabbing}
mmmm\=\kill
{\tt (compile-tf-stream \var{stream}
                        \&optional
                        \var{input-domain}
                        \var{check-only})} \\
    \>\MapsTo \=\var{revised-domain-information}, \\
    \>        \>\var{error-count}, \\
    \>        \>\var{warning-count}
\end{tabbing}

\noindent
{\tt Compile-tf-stream} takes three parameters and returns three
results.  The parameters are:

\begin{enumerate}
\item A character input stream containing TF forms followed
      by an end-of-file.
\item A domain information structure or {\tt nil}.  A structure is
      given when a domain description is being constructed in steps.
      It contains an earlier result constructed by the TF compiler 
      and is used as a base for further compilation.
\item An optional flag that is {\em true\/} if the compiler should
      operate in syntax-checking mode and {\em false\/} (the default)
      otherwise.
\end{enumerate}

\noindent
The results are:

\begin{enumerate}
\item A domain information structure.
\item The number of errors found.
\item The number of warnings found.
% \item A description of what has been deleted, altered or added.
\end{enumerate}

The \var{input-domain} is never modified.  If the result will contain
different information, a new domain structure is created instead. 
If the number of errors is nonzero, any new domain is discarded and
the input domain is returned as the result.


% \subsection{Overall structure of the compiler}

% Three passes: parsing, domain construction, domain analysis.

% The TF compiler could be considered a three-pass compiler.  The TF
% language does not require that many passes, but they provide a more
% modular structure for the code.  The ``passes'' are as follows.

%


\subsection{Scanner}

\subsubsection{Token generators}

The TF scanner is implemented as a token generator: a function of no
arguments that returns a new token each time it is called.  If there
are no more tokens, it returns the value of the variable {\tt *end-token*}.
From the point of view of the parser, any token generator can be used.
For instance, when debugging the parser it may be useful to use a
generator that returns successive elements from a list without doing
any scanning at all.\footnote{For an example, see \S\ref{list-gen-example}.}
The standard TF scanner is a generator that takes tokens from a
character input stream and employs a syntax appropriate for the TF
language.  It is created by calling the function {\tt tf-tokens}:

\begin{tabbing}
{\tt (tf-tokens \var{stream})} \MapsTo \var{generator}
\end{tabbing}

\subsubsection{Tokens and character classes}

The TF scanner tries to use a natural mapping from TF tokens to Lisp
data structures.  For instance \meta{number}s are returned as numbers
and \meta{text\_string}s as strings.  Symbols are used for several
types of tokens: for \meta{name}s; for {\em operators\/} such as {\tt +},
{\tt <=}, and {\tt --->}; and for {\em punctuation\/} characters such
as comma.

When forming symbols, the idea is to employ a natural grouping of
characters.  This is defined in terms of three disjoint {\em character
classes}:

\begin{enumerate}
\item A set of {\em name characters\/} that make up normal identifiers,
      i.e., \meta{name}s.
\item A set of {\em operator characters} that make up special
      identifiers (operators) which may consist of more than
      one character.
\item A set of {\em punctuation characters} that result in
      single-character identifiers.
\end{enumerate}

It's important that these sets are disjoint, so that the scanner can
find the boundaries of tokens without requiring that they be separated
by ``white space''.  For characters in the name or operator classes, a
maximal sequence of adjacent characters in the same set is combined
into a single token.  Characters in the punctuation set are returned
as single-character tokens.  Thus in ``\verb|apple>=pie;|'', the
sequence of tokens is ``{\tt apple}'', ``{\tt =>}'', ``{\tt pie}'',
and ``\verb|;|''.  % /\/ maybe omit the quotes here.

It's usually clear whether a character should be an operator
character or punctuation.  For example, ``{\tt -}'' and ``{\tt =}''
have to be operator characters because they can occur in
multi-character operators, and comma should be punctuation because
it should always be treated as a single character.  Characters
such as ``{\tt +}'' and ``{\tt *}'' do not occur in any
multi-character operators presently in TF, but for consistency
it makes sense to put them in the same class as ``{\tt -}''.
Since both operators and punctuation are represented as symbols,
however, moving a character from one class to the other changes
only the grouping rule without changing the type of Lisp object
that's returned.

This approach to operators has several advantages.
Multi-character operators such as {\tt >=} and {\tt --->} can be
returned as single tokens; the grouping rules are simple and do not
require that the scanner perform a context-dependent analysis or be
told what operators are in the language; and it's usually easy to
move a character from one class to another.  

However, it has the disadvantage that it's necessary to use white
space or punctuation characters to delimit adjacent operators.
Fortunately, adjacent operators are rare.  In TF, the only place where
they might occur is in arithmetic expressions.  For instance, someone
might write ``{\tt 3*-4}'' to multiply $3$ by $-4$.  Since ``{\tt
*-}'' will be returned as a single token, the user would have to
write ``{\tt 3 * -4}'' or ``{\tt 3*(-4)}'' instead.  This particular
instance of the problem could be remedied by moving ``{\tt *}'' to
the punctuation class, but that would still leave cases such as 
``{\tt 3--4}''.

\subsubsection{Readtables}

The TF scanner assigns meanings to characters by using a readtable.
The next token is read from a stream by calling {\tt read} with
{\tt *readtable*} bound to a ``TF readtable''. The TF scanner
uses the TF readtable that is the value of {\tt *tf-readtable*}.

The TF readtable allows names to be read as symbols in the usual way.
Letters are converted to upper case, just as they are in normal Common
Lisp input.  As far as the readtable is concerned, the characters
allowed in names are defined implicitly as the characters Common Lisp
treats as ``constituents'' minus any characters defined as operator
or punctuation characters or explicitly defined as illegal.

Tokens other than names are read by character macros.  The character
macros are defined according to character classes which include
the classes described above plus some others such as {\em digit}.  All
macro characters in a class are given the same function as their macro
definition, and this function is responsible for reading the entire
token.  For instance, all digits are defined to call a function that
reads a number, and all operator characters are defined to call a
function that reads a sequence of characters in the operator class and
assembles them into a symbol.  Some characters do not fit into any
class and are given special macros of their own.

The most important special cases are ``{\tt ;}'' and ``{\tt ?}''.

Semicolon is a special case so that it can handle comments.
A single semicolon behaves like a punctuation character and
returns the symbol ``{\tt ;}''.  However, the function that
handles semicolons also looks ahead to see if the first
semicolon is followed by two others.  In that case, it reads
the rest of the line as a comment.

Question mark is a special case so that {\tt question-reader}, the
OBase routine for reading \meta{actor\_restriction}s, can also be used
by the TF scanner.  The macro function for ``{\tt ?}'' calls
{\tt question-reader} and whatever results is returned as a
single token.  The TF compiler then uses that token as-is.

However, {\tt question-reader} expects some characters to have
different meanings than they do in the TF readtable.  For instance,
it expects that ``\verb|{|'' will be defined as a character macro
that reads a list ending in ``\verb|}|'', but in the TF readtable
``\verb|{|'' is a punctuation character.  Consequently, the scanner's
macro for ``\verb|?|'' establishes a different readtable before
calling {\tt question-reader}.  This readtable is constructed by
calling the function

\begin{tabbing}
{\tt (make-actor-readtable \var{base-readtable})}
        \MapsTo \var{actor-readtable}
\end{tabbing}

The base readtable can be any readtable, but the scanner uses
the TF readtable as the base.  {\tt Make-actor-readtable}
redefines characters in a copy of the base readtable to create a
readtable suitable for use inside \meta{actor-restriction}s.

% {\tt question-reader} is the macro function for ``{\tt ?}'' in this
% new readtable, and other characters, such as ``\verb|{|'', also
% have the required definitions.

This example shows that it is possible to change the low-level syntax
of the language within a well-defined context.  However, it's important
to bear in mind that the compiler has to be able to tell when it's time
to switch readtables.  In this case, the ``{\tt ?}'' character marks
the start of the new context, and the syntax of \meta{actor-restriction}s 
is such that there is always a well-defined end.  In other cases,
the compiler might not be able to tell when to switch readtables,
and we would have to change the TF language so that it could tell.
For example, consider a {\tt has}-actor such as \verb|?{has length 3}|.
The general syntax is as follows:

\begin{quote}
\begin{tabbing}
{\tt ?\{has \meta{function name} \meta{argument} \ldots \meta{result}\}}
\end{tabbing}
\end{quote}

If the readtable had to be changed between the last \meta{argument}
and the \meta{result}, we'd have to introduce some punctuation or
other token to mark the end of the arguments.  Otherwise, we couldn't
tell the \meta{result} was a \meta{result}, rather than another
\meta{argument}, without reading it to see if it was followed by
``{\tt \}}'.  That is, in order to tell that the expression was a
\meta{result}, we'd already have to know how to read a \meta{result}.


% \subsubsection{Numbers and dot-operators}


\subsection{Parser}

This section describes the parser and how it is derived from the
grammar for the TF language.  It also explains some of the coding
conventions and briefly describes a set of auxiliary procedures that
are used for such things as recognizing tokens, reporting errors, and
error recovery.

Recursive descent parsers can be written almost directly from
LL(1) grammars.
A nonterminal symbol in the grammar is implemented as a procedure that
consumes one or more tokens and then returns some result, such as a
parse tree.  For convenience, we will also refer to these procedures as
``nonterminals''.

For a grammar to be LL(1), there must not be any left-recursive
productions, and it must always be possible to select between
alternatives by looking only at the current token.  The published
grammar for the TF language is not quite LL(1), but it is still
possible to write a parser by using a somewhat looser correspondence
between grammar rules and procedures.  It is also possible for a
recursive descent parser to handle some cases that cannot be expressed
in an LL(1) grammar at all.


\subsubsection{Interface to the scanner}

The scanner provides a token generator which is accessed from within
the parser by calling the procedures {\tt token} and {\tt next-token}.
{\tt Token} returns the current token; {\tt next-token} advances to
and returns the next token by calling the token generator, making the
result the new current token.  

There is also a facility for looking ahead one or more tokens which is
used at present only when a number is followed by a token that begins
with ``{\tt .}''.  {\tt (Push-token \var{token})} makes the next call to
{\tt next-token} return \var{token}.  When a number is followed by an
operator such as ``{\tt ..}'', the scanner returns the number after
calling {\tt push-token} to save the operator.  The next call to {\tt
next-token} will then return the operator.

It is important to have a convention for when the parser advances
to the next token.  In the TF parser, the following rules apply,
and any departure from them is explicitly noted.

\begin{itemize}
\item When called, a nonterminal can assume that the current token
      has already been set to the first token it should consider.
\item A nonterminal matches an initial segment of the input tokens.
      On exit, the current token should be the first one that is
      not part of the segment matched.  That is, it should be the
      first token that the rest of the parser should consider.
\end{itemize}


\subsubsection{Parsing utilities}

In most cases, {\tt token} and {\tt next-token} are not called
directly.  Instead, one of the following procedures is used.

\begin{itemize}

\item {\tt (token-is \var{token})} \MapsTo \bool

{\tt Token-is} compares its argument with the current token.  If they
are the same ({\tt eql}), it advances to the next token by calling
{\tt next-token} and returns {\em true}.  Otherwise it returns
{\em false} without calling {\tt next-token}.

\item {\tt (token-satisfies \var{predicate})} \MapsTo \var{token}

{\tt Token-satisfies} is used to determine whether the current token
satisfies a predicate.  If it does, {\tt token-satisfies} advances to
the next token by calling {\tt next-token} then returns the token that
was tested.  Otherwise, {\tt token-satisfies} returns {\em false}
without calling {\tt next-token}.  The {\em predicate} is a function
of one argument, a token, and returns \bool.

\item {\tt (must-be \var{token})} \MapsTo \var{token}

{\tt Must-be} is similar to {\tt token-is} but requires that the current
token be a particular token.  If the current token is not acceptable,
a syntax error will be reported.

\item {\tt (must-satisfy \var{predicate} \var{description})}
         \MapsTo \var{token}

{\tt Must-satisfy} is similar to {\tt token-satisfies} but requires
that the current token satisfy the predicate.  If the current token is
not acceptable, a syntax error will be reported.  The {\em description}
is a string that will be used in the error message.  For example:

\verb|(must-satisfy #'numberp "a number")|

\item {\tt (one-or-more \var{nonterminal}
              \&key :separator \var{token}
                    :until \var{end-token})}
      \MapsTo \var{list}

{\em Nonterminal} is a procedure that parses a nonterminal in the
grammar.  {\tt One-or-more} parses a sequence of one or more
occurrences of the nonterminal until a particular {\em end-token\/}
appears.  The {\em separator\/} is optional.  {\tt One-or-more}
returns a list of the results returned by {\em nonterminal}.
                    
\item {\tt (zero-or-more \var{nonterminal}
              \&key :separator \var{token}
                    :until \var{end-token})}
      \MapsTo \var{list}

This function is similar to {\tt one-or-more} but allows zero
occurrences of the nonterminal.
                    
\end{itemize}


\subsubsection{Error recovery}

The compiler uses a version of the ``S-algol error recovery scheme''
described in the book {\em Recursive Descent Compiling\/} by A. J. T.
Davie and R.  Morrison (Ellis Horwood, 1981), which in turn comes from
a scheme developed by D. A. Turner for a SASL compiler.
%
Some additional conventions are required by the error recovery
scheme:

\begin{itemize}
\item If a syntax error is detected, it should be signalled by
      calling the {\tt syntax-error} procedure.  That call will
      (usually) return, after which parsing should continue as
      much as possible as if the syntax had been correct.
\item At a point where a certain token must be present, require that
      it be present by calling the the procedure {\tt must-be}.
      Something similar can be done for classes of tokens by calling
      {\tt must-satisfy}.
      The ``{\tt must-}'' procedures call {\tt syntax-error}
      automatically.
\end{itemize}

The error recovery scheme is very simple, but it works reasonably well
in a number of common cases.  The basic idea is to take advantage of
points where a particular token or class of tokens must occur.  If the
current token is not acceptable, the normal action would be to report
an error.  However, if the parser is already in an error state it can
try to recover instead, by discarding tokens until an acceptable one
appears.  While the parser is waiting to recover from an error,
further error messages are suppressed


\subsubsection{Blocking tokens}

Since the error recovery scheme can sometimes ``eat'' too much of
the source text, it is possible to specify a list of {\em blocking
tokens\/} that it will not pass.  The {\tt one-or-more} and
{\tt zero-or-more} procedures will also respect this list when
they examine the current token to see whether their iteration
should continue.\footnote{Note that this does not mean a blocking
token will {\em never\/} be skipped inappropriately.  Since they
are not reserved words, a nonterminal might interpret them in other 
ways or mistake them for something else.}

A list of blocking tokens can be specified in two ways: by
using the {\tt :blocking-tokens} keyword parameter to {\tt zero-}
or {\tt one-or-more}, or by binding {\tt *blocking-tokens*}.
The main use of blocking tokens in the TF compiler is when
parsing a schema, when the blocking tokens are the symbols 
that can begin schema clauses (e.g. {\tt expands}, {\tt conditions})
or end schemas.


\subsubsection{Translating grammar rules to Lisp code}

To illustrate the coding conventions, we'll consider a subset of
LL(1) grammars that can be translated directly to code and that is
sufficient to cover most of the statement-level cases that arise.

A given nonterminal may be defined by more than one rule, or by
one rule with several alternatives.  If so, we need to be able to
select among the alternatives without looking beyond the current
token.  It will clearly be possible to do this if all but at most
one of the alternatives begins with a terminal symbol and the
remaining alternative (if there is one) must begin in some other
way.  For instance, consider a definition of \meta{statement} in
which every alternative except the assignment statement begins with
a reserved word.  By looking at only the current token, we can
determine what kind of statement we have.  If the token isn't one
of the reserved words, we have an assignment.

A rule of this form can be translated into a function definition as
follows\footnotemark (or by using the equivalent {\tt token-case}
macro which inserts the calls to {\tt next-token} automatically):

% /\/: Should describe token-case earlier.  At present, it's not
%      really described at all.

\footnotetext{When describing such translations, we'll present grammar
rules {\em abstractly\/} and then show the corresponding Lisp.  To do
this, we need notations that stand for the terminal and nonterminal
symbols that would actually appear in a rule, and here they are:
upper-case italic names in angle-brackets such as \meta{NT} stand for
nonterminal symbols, and italic names without brackets such as
\var{t$_1$} or, later, \var{separator}, stand for terminal symbols.}

\begin{quote}{\tt
\begin{tabbing}
mmm\=\kill
  \meta{NT} \BNFdef\ \var{t$_1$} \meta{NT$_1$}
             \ordef\ \var{t$_2$} \meta{NT$_2$}
             \ordef\ \ldots\ 
             \ordef\ \meta{NT$_n$} \\
    \>\expandsto
      (d\=efun \meta{NT} () \\
    \>  \>(c\=ase (token) \\
    \>  \>  \>((\var{t$_1$}) (next-token) (\meta{NT$_1$})) \\
    \>  \>  \>((\var{t$_2$}) (next-token) (\meta{NT$_2$})) \\
    \>  \>  \>\ldots \\
    \>  \>  \>(otherwise \meta{NT$_n$})))
\end{tabbing}}
\end{quote}

This can clearly be generalized to handle cases where the alternatives
are more complex, and to cases where arbitrary predicates (not just
{\tt eql}) are applied to the current token.  In other cases, it will
be necessary to (implicitly) transform the grammar to a form that can
be handled directly. % or to depart somewhat from LL(1).

Once we've selected an alternative, we're committed to it and
should report an error if the input doesn't match what we expect.
Some of the more common constructs that occur within alternatives
can therefore be translated as follows:

\begin{quote}{\tt
\begin{tabbing}
mmm\=\kill
  \meta{NT} \\
     \>\expandsto (\meta{NT})\\
\\
  \var{literal} \\
     \>\expandsto (must-be '\var{literal}) \\
\\
   \optional{\var{literal} \meta{NT}} \\
     \>\expandsto (when (token-is '\var{literal}) (\meta{NT})) \\
\\
   \meta{NT} \arbno{\optional{\var{separator} \meta{NT}}} \var{end} \\
     \>\expandsto
       (one-or-more \verb|#'|\meta{NT} 
                    :separator '\var{separator}
                    :until '\var{end})
\end{tabbing}}
\end{quote}

Note that in the last case, both instances of \meta{NT} must be
replaced by the same nonterminal.  As a concrete example, consider
the following incomplete grammar for a fragment of the TF language:

\begin{Lindent}

\begin{syntaxtable}
\meta{tf form} 
   & \BNFdef & {\tt always} \meta{always form} \\
   & \ordef  & {\tt types} \meta{types form}   \\
\end{syntaxtable}

\begin{syntaxtable}

\meta{always form}
   & \BNFdef & \meta{pattern assignment} 
               \arbno{\optional{{\tt , }\meta{pattern assignment}}}
               {\tt ;} \\

\meta{pattern assignment}
   & \BNFdef & \meta{pattern} \optional{{\tt =} \meta{value}} \\

\meta{types form} 
   & \BNFdef & \meta{type definition} 
               \arbno{\optional{{\tt , }\meta{type definition}}}
               {\tt ;} \\

\meta{type definition} 
   & \BNFdef & \meta{name} {\tt =} \meta{name set}

\end{syntaxtable}

\end{Lindent}

The corresponding procedures are shown below.  They are written using
the convention that the procedure for a nonterminal has a name in
angle-brackets.  Note that only the parsing code is shown, without
anything that would construct a result.

\begin{Lindent}
\begin{verbatim}
(defun <tf-form> ()
  (token-case
    ((always) (<always-form>))
    ((types)  (<types-form>))
    (t (syntax-error "Unknown TF form."))))

(defun <always-form> ()
  (one-or-more #'<pattern-assignment> :separator '|,| :until '|;|))

(defun <pattern-assignment> ()
  (<pattern>)
  (when (token-is '=)
    (<value>)))

(defun <types-form> ()
  (one-or-more #'<type-definition> :separator '|,| :until '|;|))

(defun <type-definition> ()
  (<name>)
  (must-be '=)
  (<name-set>))
\end{verbatim}
\end{Lindent}

Of course, grammars are not always written in so accommodating a style.
For instance, \meta{types form} might have been defined as follows:
%
\begin{Lindent}
\begin{syntaxtable}

\meta{types form} & \BNFdef & \meta{types body}{\tt ;} \\

\meta{types body} 
   & \BNFdef & \meta{type definition} \\
   & \ordef  & \meta{types body}{\tt , }\meta{type definition}

\end{syntaxtable}
\end{Lindent}
%
With this definition, some analysis is needed before it becomes clear
that a \meta{types form} is a sequence of \meta{type definition}s
separated by commas and terminated by a semicolon.

This example also serves to illustrate a difficult case for us,
because it involves left-recursion.  Some analysis and recasting
is therefore necessary.  In other cases, it may not be
necessary---because we can write out the recursive procedures
rather than using {\tt one-or-more}---but it may still be
desirable.  Indeed, most grammars not written specifically for
use with a recursive descent parser will require some modification
hare and there.  Whether you handle this by explicitly rewriting
the grammar or just by translating it more loosely into Lisp
is up to you.


\subsubsection{Testing}

Several procedures are available for testing the parser or for
parsing fragments of TF.  The most important ones are:

\begin{itemize}

\item {\tt (compile-tf-string \var{nonterminal} \var{string})}
         \MapsTo \var{object}, \var{error-count}

This creates a TF scanner for a string input stream and calls
{\em nonterminal}.  The results are whatever object {\em nonterminal\/}
returns and the number of errors that occurred.

\item {\tt (compile-tf \var{nonterminal} \var{generator})}
         \MapsTo \var{object}, \var{error-count}, \var{warning-count}

{\tt Compile-tf} uses the {\em generator\/} as the scanner and calls
{\em nonterminal}.  The results are whatever object {\em nonterminal\/}
returns and the number of errors and warnings that occurred.

\item {\tt (list-tokens \var{list})} \MapsTo \var{generator}

{\tt List-tokens} creates a generator that returns successive
elements of a list followed by the value of {\tt *end-token*}

\end{itemize}

\noindent
Here's an example:

\label{list-gen-example}
\begin{verbatim}
(compile-tf #'<pattern>
            (list-tokens '(|{| on a b |}|)))
\end{verbatim}

\noindent
returns the list {\tt (on a b)}.


\subsection{Compile-time analysis}

\subsubsection{Levels}

The level algorithm assigns levels to actions, then to effects.
Conditions are then checked against effects.  Actions, effects,
and conditions are represented by the first word of their
respective patterns.

The level number of an effect is the max of the level numbers
of all actions that have that effect.

The level number of an action is determined as follows:

\begin{enumerate}

  \item

     Construct a directed graph $G$ in which the vertices are actions
     and there's an edge from $A$ to $A'$ whenever $A'$ is part of an
     expansion of $A$ or can be brought in by an {\tt achieve} / 
     {\tt only\_use\_for\_effects} link.

  \item

     Construct a graph, $SCC(G)$, of the strongly connected components
     of $G$.  Two actions $A$ and $A'$ are in the same component if there's
     both a path from $A$ to $A'$ and one from $A'$ to $A$.  There's an edge
     in $SCC(G)$ from component $C$ to component $C'$ if there's an edge
     in $G$ from an action in $C$ to an action in $C'$.

  \item

     Starting from actions that correspond to task schemas, determine
     the length of the longest path to each component in $SCC(G)$,
     giving each edge length $1$.
     The length of the longest path to a component becomes the
     level number of all actions in that component.

\end{enumerate}

\end{document}
